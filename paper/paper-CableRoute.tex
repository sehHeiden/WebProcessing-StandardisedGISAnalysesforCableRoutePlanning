% !TeX spellcheck = en_US

\documentclass[acmtog]{acmart}
\usepackage[acronym]{glossaries}
\usepackage{todonotes}

\AtBeginDocument{%
	\providecommand\BibTeX{{%
			Bib\TeX}}}

\setcopyright{acmcopyright}
\copyrightyear{2022}
\acmYear{2022}
\acmDOI{XXXXXXX.XXXXXXX}




%%
%% These commands are for a JOURNAL article.
\acmJournal{JACM}
\acmVolume{37}
\acmNumber{4}
\acmArticle{111}
\acmMonth{8}

\newacronym{wfs}{wfs}{Web Feature Service}
\newacronym{wps}{wps}{Web Processing Service}

\begin{document}
	\title{Web Processing - Standardised GIS Analyses for Cable Route Planning}
	
	\author{Sebastian Heiden}
	\email{u38439@hs-harz.de}
	\affiliation{%
		\institution{Harz University of Applied Sciences}
		\streetaddress{Friedrichstrasse 57-59}
		\city{Wernigerode}
		\state{Saxony-Anhalt}
		\country{Germany}
		\postcode{38855}
	}
	
	
	\renewcommand{\shortauthors}{Heiden}
	
	\begin{abstract}
		add as last part
	\end{abstract}
	
	%%
	%% The code below is generated by the tool at http://dl.acm.org/ccs.cfm.
	%% Please copy and paste the code instead of the example below.
	%%
	\begin{CCSXML}
		<ccs2012>
		<concept>
		<concept_id>10010520.10010553.10010562</concept_id>
		<concept_desc>Computer systems organization~Embedded systems</concept_desc>
		<concept_significance>500</concept_significance>
		</concept>
		<concept>
		<concept_id>10010520.10010575.10010755</concept_id>
		<concept_desc>Computer systems organization~Redundancy</concept_desc>
		<concept_significance>300</concept_significance>
		</concept>
		<concept>
		<concept_id>10010520.10010553.10010554</concept_id>
		<concept_desc>Computer systems organization~Robotics</concept_desc>
		<concept_significance>100</concept_significance>
		</concept>
		<concept>
		<concept_id>10003033.10003083.10003095</concept_id>
		<concept_desc>Networks~Network reliability</concept_desc>
		<concept_significance>100</concept_significance>
		</concept>
		</ccs2012>
	\end{CCSXML}
	
	\ccsdesc[500]{Computer systems organization~Embedded systems}
	\ccsdesc[300]{Computer systems organization~Redundancy}
	\ccsdesc{Computer systems organization~Robotics}
	\ccsdesc[100]{Networks~Network reliability}
	
	%%
	%% Keywords. The author(s) should pick words that accurately describe
	%% the work being presented. Separate the keywords with commas.
	\keywords{datasets, neural networks, gaze detection, text tagging}
	
	\received{20 February 2007}
	\received[revised]{12 March 2009}
	\received[accepted]{5 June 2009}
	
	%%
	%% This command processes the author and affiliation and title
	%% information and builds the first part of the formatted document.
	\maketitle
	
	\section{Introduction}
	
	Sometimes, finding the shortest path is not sufficient. Additional parameters play also have to be taken into consideration. As the steepness of a road or the soil,
	play an important role for the building cost of a road or pipeline \todo{source}. For planing the additional routes for a power grid, additional aspects as legal regulations and acceptance by the local population have to be taken in consideration. Also technical aspects, as the effects on the grid stability might be further points to consider.\cite{schafer_understanding_2022}
	\todo{what is the least cost path}


	\section{Methods}
	We retrieve a set of different spacial data-sets from  public sources as a basis to create the cost raster. Field of study are the counties of Cuxhaven and Osterholz in the state of Lower Saxony, Germany.
	Areas protected by different European and National conservation laws are provided by the German Environment Agency as \acrfull{wfs}~\cite{noauthor_schutzgebiete_2015}. The nation wide land coverage (ATKIS) with a scale of 1:250000 are provided by the Federal Agency for Cartography and Geodesy\cite{noauthor_digitales_2021}. The nation wide power grid (tags: 'power': line) has been retrieved via OpenStreetMap.\cite{boeing_osmnx_2017}
	Local data as houses at Level of Detail 1 are offered by the State Office for Geoinformation and Land Surveying of Lower Saxony\cite{noauthor_opengeodatani_2022}. In addition local planning geodata for the land usage are taken from 
	from 'Metropolplaner' (Planing data Lower Saxony \& Bremen)\cite{noauthor_metropolplaner_2022}
	
	PyWPS\cite{noauthor_welcome_2016} is used to offer the least cost path algorithm as a \acrfull{wps}.
	The for the initial implementation of the least cost path algorithm the implementation for the QGIS-Plugin 'Least Cost Path'\cite{noauthor_leastcostpathdijkstra_algorithmpy_2022} has been taken into account.
	
	The different layers from the different entities are optionally filtered, buffered and than converted into rasters. Filtering the layers of the files for special attributes enables to further differentiate further. For examples makes it possible to differentiate between heath and uncultivated land.
	 Adding a buffer can be used either used to convert a line objects as power grids and streets into a polygon with the correct physical width, or to add minimum distance from an existing of planed area to the new power grid.
 	Each of theses rasterized layers are given a weight, or cost that expresses the cost of using land covered by this layer.
 	The costs of all layers of all layers is aggregated with the maximum function. Thus, an area covered by multiple layers is uniformly used with the highest costs. For every area in the study area. that is not covered by any layer, is given the default cost.
	The costs has been grouped into different levels~(see table \ref{table:1}) starting from preferential areas with very low costs, via no restrictions, which is the default, used when no other layers are covering the local area, to restricted, strongly restricted and prohibited areas with high costs. These higher costs resemble the degree how much a local area should be avoided, while routing the path. The ratio of the higher costs to the lower costs directly translates into the additional diversion in pixels the algorithm is willing to go, for avoiding an area of high costs.
	Thus, as prohibited areas describe a legal obligation, not to use these areas or only to the utmost minimum, the weight that resembles the costs for these types of areas, has to be especially high.
	\todo{all touched}
	
	\begin{table}[h!]
		\caption{Used levels of costs, the applied numerical equivalent and example layer this cost have been used for.}
		\label{table:1}
		\centering
		\begin{tabular}{ l  r  l }
			Cost Level & Cost &  Example\\
			\hline
			Prohibited & 500					& National Parks, Buildings \\
			strongly Restricted & 10 	& Bird Reserve \\
			Restricted & 5	& Industrial Areas \\
			No Restriction & 0.5					& Default\\
			Preferential & 0.1					& Power Grid, Motorway Buffers\\
		\end{tabular}
	\end{table}
	The completed list of layers and the processing applied to them, can be found in Supplement S1. \todo{create the Supplement from the processing rules}

	\section{Results}
	In this chapter we want to show the different cost rasters, that were created from the very same set of layers, but computed for different resolutions. From this different rasters the cost paths are calculated and compared. In the last step the rasters with lower resolution were used to calculate in a way, that they shall result in similar paths as if the paths were computed from a high resolution raster.
	\subsection{Cost Rasters}
	\subsection{Least Cost Paths}
	
	\begin{table}[h!]
	\caption{Least cost paths for the different resolutions of the rasters.}
	\label{table:2}
	\centering
	\begin{tabular}{ r  r  r r }
		resolution /m & $length_{all touched false} /m$ & $length_{all touched true} /m$ & mean minimum distance /m\\
		\hline
		5 & 76136.27					& 78002.00 & 126.04 \\
		10 & 75430.10 	& 77936.57 & 277.92 \\
		50 & 76135.02	& 70619,95 & 1140.01 \\
		100 & 76283.80	& 74120.73				& 1946.41 \\
	\end{tabular}
	\end{table}
	\subsection{Faster Processing of the Cost Path Algorithm}
	
	The paths between the all\_touched False runs is less distinct. The mean average distance between the 100 m resolution run and the 5 m resolution run is 257.97 m. \todo{add the overall cost}
	
	\begin{table}[h!]
		\caption{Ratio Category values of each least cost path.}
		\label{table:3}
		\centering
		\begin{tabular}{ r  r  r  r  r  r  r }
			resolution /m & all touched & $ r_{Preferential} \% $  & $ r_{No Restriction} \% $ & $ r_{Restricted} \% $ & $ r_{strongly Restricted}\% $ & $ r_{Prohibited} \% $ \\
			\hline
			5 & False &  4.29 & 59.01  & 8.88 & 0.73 & 27.08  \\
			5 & True &  17.38 & 67.25  & 1.17 & 0.97 & 13.23\\
			
			10 & False &  17.90 & 68.73  & 0.92 & 0.80 & 11.64\\
			10 & True &  17.37 & 66.42  & 1.47 & 1.38 & 13.36\\
			
			50 & False &  18.73 & 67.87  & 0.87 & 0.68 & 11.85\\
			50 & True &  8.58 & 74.95  & 3.95 & 4.08 & 8.45\\
			
			100 & False &  18.48 & 69.82  & 1.31 & 0.90 & 9.52\\
			100 & True &  6.16 & 71.74  & 5.75 & 9.69 & 7.04\\			

		\end{tabular}
	\end{table}
	\todo{Check why values are so different for 5 m all touched False.}
	\todo{Discuss WHY}
	\subsection{Faster Processing of the Cost Path Algorithm}
	
	
	\section{Discussion}
	\todo{discuss aggregation with max vs sum}
	\section{Related Works}
	\section{Conclusion}


	

	





\begin{acks}
	\ldots
\end{acks}

%%
%% The next two lines define the bibliography style to be used, and
%% the bibliography file.
\bibliographystyle{ACM-Reference-Format}
\bibliography{Web_Processing.bib}


%%
%% If your work has an appendix, this is the place to put it.
\appendix

\section{Research Methods}

\subsection{Part One}



\subsection{Part Two}



\section{Online Resources}



\end{document}
\endinput
%%
%% End of file `sample-acmsmall.tex'.